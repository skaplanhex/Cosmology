%\documentclass[aps,preprint,prl]{revtex4-1}
\documentclass{article}
\usepackage{graphicx}  % needed for figures
\usepackage{dcolumn}   % needed for some tables
\usepackage{bm}        % for math
\usepackage{amssymb}   % for math
\usepackage{url}
\usepackage{hyperref}
\usepackage{fullpage}
\renewcommand\refname{Some Main References} % change from 'References' as the title of the bibliography

\begin{document}
\title{Review of Dark Matter Searches By Particle Physics Experiments}
\author{Steven M Kaplan\\Department of Physics and Astronomy\\Rutgers, The State University of New Jersey}
\date{}

\maketitle
%\nocite{*} % make the entire reference list even though the references haven't been cited
\nocite{cmsMonojet,cmsGammaPlusMET,BertoneBook,Buckley2011216,PhysRevD.33.3495} %Just show these sources for now
My paper will serve as a review of the results of dark matter searches; in particular, those done by particle physics experiments such as (but not necessarily limited to) CMS, ATLAS, DAMA, PAMELA, CDMS, CoGeNT, and XENON.  Clearly the entirety of dark matter research (even when limited to particle physics experiments) cannot be summarized in a paper such as this, but a sincere attempt will be made to be fairly inclusive.  While the scope will be more broad, care will be given to provide salient details on each experiment.  My goal is, in general, to bring the reader up to date on the results of these experiments.  I will also touch on the conflicts between some of the results (\emph{i.e.} possible signals seen in regions that previous experiments have excluded).  Another goal is to include some material on the plans of future experiments.
\\ \\
The review will begin with some words on what evidence we have for the existence of dark matter (galactic rotation curves, measurements of $\Omega_{\mathrm{matter}}$, anisotropies in the cosmic microwave background (CMB), etc.).  Next will be a discussion of some Beyond Standard Model (BSM) dark matter candidates (\emph{i.e.} supersymmetry (SUSY)).  Afterward, I will present the basic physics behind how the various dark matter particle detectors work, the analysis procedure in each case, and their results (given in context).  Finally, I will discuss future plans of these and other detectors.

\bibliographystyle{hunsrt}
\bibliography{proposal}
\end{document}