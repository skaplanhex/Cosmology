\documentclass[aps,reprint,prl]{revtex4-1}
\usepackage{graphicx}  % needed for figures
\usepackage{dcolumn}   % needed for some tables
\usepackage{bm}        % for math
\usepackage{amssymb}   % for math
\usepackage{url}
\usepackage{hyperref}

\begin{document}
\title{Inflation As A Solution To Problems With the Hot Big-Bang Model}
\author{Steven Kaplan}
\affiliation{Department of Physics and Astronomy, Rutgers, The State University of New Jersey, 136 Frelinghuysen Road, Piscataway, New Jersey 08854, USA}

\begin{abstract}
Before the proposition that our universe underwent a period of very rapid inflation, there were numerous outstanding issues with the hot big-bang model of cosmology.  These problems are the well known horizon problem, flatness problem, the fact that we've never observed a magnetic monopole, the lack of source of perturbations in the CMB and matter density, and why the cosmological constant $\Lambda$ is non-zero.  All of the problems will be discussed, with an explanation of whether or not cosmic inflation solves any of them.
\end{abstract}

\maketitle
\section*{Introduction}
The hot big-bang model is a very successful explanation of the history of our universe in many ways.  It fits our observations that the universe is not only expanding, but its expansion is accelerating.  It explains the Cosmic Microwave Background (CMB) radiation we observe, and it predicts the abundances of light elements after nucleosynthesis that we can observe today.  There are, however, some problems with the model that need to be addressed.  In this paper, I discuss five such problems.
\section*{Horizon Problem}
It is convenient in many cases in cosmology to use the concept of conformal time
\begin{equation}
\eta=\int_0^t\frac{dt'}{a(t')}\;.
\end{equation}
The conformal time defines a particle horizon whose comoving distance to any given observer is equal to the maximum distance that a particle could travel in the age of the universe to get to the observer \cite{wiki:particlehorizon}.  Past this horizon, it is impossible that a particle could ever have interacted with anything at our location.  Now, one of the bedrock principles of cosmology is that on a large enough scale, the universe is homogeneous and isotropic.  One such support for this is the Cosmic Microwave Background (CMB) radiation consisting of decoupled photons after recombination \cite{wiki:recombination}.  Apart from anisotropies on the $\mu$K scale, the CMB has shown to be quite uniform.  Given the idea of the particle horizon, it is not clear why this should be.  In order for the CMB to be uniform in temperature, photons must have travelled across the entire universe in order for them to be close enough to one another to reach thermal equilibrium.  This is entirely inconsistent with the concept of the particle horizon as explained above.  Consider photons from the CMB reaching an observer from directions 180$^\circ$ apart.  Recombination happened only  $\sim$378,000 years from the big bang \cite{wiki:recombination}, so it is sufficient to assume that photons coming from the last scattering surface (LSS) have travelled a conformal time $\eta_0$, the conformal time today.  The fact that these photons are just meeting near the observer now and are both coming from opposite sides of the universe would imply they had never met in the past (see Figure 1).  More quantitately, Guth \cite{guth1981} has shown that the ratio of the volume of the universe in which the photon could have travelled to the total size of the universe is $\sim10^{-83}$.  If the photons were never able to meet throughout the history of the universe, how is it possible for them to be in thermal equilibrium?
\begin{figure}[h]
\includegraphics[width=0.5\textwidth]{dodelson62cropped}
\caption{This is figure 6.2 from \cite{Dodelson}.  The interior of the cone represents the interior of the particle horizon.  Note that the volume increases with time.  It shows how photons coming from the last scattering surface that are just reaching us now could not have been in causal contact with each other before reaching us.}
\end{figure}
\section*{Flatness Problem}
This discussion mainly follows \cite{masters,wiki:flatness}. This problem is not a seeming contradiction in the theory as is the case with the horizon problem, but rather it is a statement that the overall density of the early universe must have been extremely fine-tuned to produce our universe as we observe it today.  This can be seen from the Friedmann equation
\begin{equation}
H^2=\frac{8\pi G}{3}\rho-\frac{k}{a^2}\;.
\end{equation}
Multiplying both sides by $3a^2/8\pi G \rho$ and collecting terms gives
\begin{equation} \label{eq:flatmain}
\left(1-\frac{1}{\Omega}\right)\rho a^2=\frac{3k}{8\pi G}\;,
\end{equation}
where I have used the fact that the critical density $\rho_c=3H^2/8\pi G$ and that $\Omega=\rho/\rho_c$.  Since the right hand side of equation \ref{eq:flatmain} is constant, the left side must be constant as well.  For most of the history of the universe, $\rho$ was dominated either by matter or radiation.  From the fact that $\rho_{matter}\propto a^{-3}$ and $\rho_{rad}\propto a^{-4}$, we can see that the decrease of the density of the universe outpaces the scale factor increase over the history of the universe.  This implies that the quantity $\rho a^2$ decreases with time.  As the left hand side is a constant, it must be the case that $\left(1-\frac{1}{\Omega}\right)$ increases with time. Since the Planck era, $\rho a^2$ has decreased by a factor of $\sim 10^{60}$, so $\left(1-\frac{1}{\Omega}\right)$ must have increased by this same amount.  The lower bound of the WMAP measurement of $\left(\Omega-1\right)$ is -0.031 \cite{WMAP}, noting that the result is consistent with zero within uncertainty, so $\left(1-\frac{1}{\Omega}\right)\simeq-0.032$.  If this quantity must have increased by a factor of $\sim 10^{60}$ since the Planck era, this means that at that time $\left(1-\frac{1}{\Omega}\right)$ must have been precise within a few parts in $10^{62}$!  This is the flatness problem.  An aside that the author cannot resist: the flatness problem somehow seems like less of a problem than the horizon problem.  One can invoke the anthropic principle and say that, indeed, humanity can only have existed with a very precise set of cosmological parameters, so it makes sense that these parameters seem finely tuned since we're here to study them.  Also, the author doesn't quite see the problem with a parameter needing to be finely tuned.  Why is it any more preferable for the density to not be as finely tuned?  Cosmology is the study of the structure and history of the universe as it is, not what we want it to be.
\section*{Monopole Problem}
Preskill, and many others as well, argue that Grand Unified Theories (GUTs) predict a very large density of magnetic monopoles in the early universe \cite{preskill}.  So much so, that Preskill claims that monopoles should ``dominate the energy density of the universe before the time of helium synthesis."  This clearly flies in the face of how we understand Big-Bang Nucleosynthesis (BBN) to have occurred.  Magnetic monopoles are stable, so the only way they could be destroyed is via annihilation with an anti-monopole.  This of course becomes less likely as the early universe rapidly expands.  If GUTs are correct in their prediction of magnetic monopoles, especially in the quantity they predict, it is a major problem that we have not observed them yet.
\section*{Perturbation and Cosmological Constant Problems}
As was discussed before, there are anisotropies in the CMB and in the matter density of our universe.  We of course know that massive objects exist on a large scale (\emph{e.g.} galaxies).  These must have been formed by some perturbation to the equilibrium state of the universe.  Where do these perturbations come from?  Also, we have a good amount of evidence from groundbreaking work \cite{riess} with high redshift supernovae that the expansion of the universe is accelerating and thus the cosmological constant $\Lambda$ in the Einstein Field Equations is non-zero.  As discussed above, we have also observed that the universe is very close to (if not exactly) flat.  This cannot be accounted for with baryonic matter, dark matter, and radiation alone, as the total density of the universe must be equal to the critical density in order to be flat.  A very possible explanation for these two observations is that there is a field of dark energy permeating the entire universe with a constant density.  The problem here is clear: we have no idea what dark energy actually is, only that it makes sense given our observations that it should exist.
\section*{Inflation}
This discussion mainly follows \cite{Dodelson}.  In figure 2, if one extrapolates the comoving horizon back in time, the conclusion reached would be that most of the contribution to the comoving horizon happened in later times in the evolution of the universe.  The assumption is, of course, that such an extrapolation is valid.  If some sort of discontinuity occurred over the interval of our extrapolation, our assumption about the main contribution to $\eta$ only occurring in later times may not be correct.  
\begin{figure}[h] \label{fig:dodelson61}
\includegraphics[width=0.5\textwidth]{dodelson61cropped}
\caption{This is figure 6.1 from \cite{Dodelson}.  It shows the comoving horizon as a function of the scale factor $a$.  Dodelson shows two comoving wavelengths (which are constant in time), and it is clear that the wavelength is larger than the horizon during the early universe.  It is not until well after recombination that the quadrupole scale is within the horizon.}
\end{figure}
It helps to recast $\eta$ as
\begin{equation}
\eta=\int_0^a\frac{da'}{a'}\frac{1}{a'H(a')}\;.
\end{equation}
Here, $1/aH$ is referred to as the comoving Hubble radius.  Through the entirety of the matter, radiation, and dark energy domination periods, the comoving Hubble radius increases with time.  The thought then occurs: what if there were a period of time in the early universe in which the comoving Hubble radius actually \emph{decreased}?  If this were the case, then it is conceivable that particles which are separated by at least the comoving Hubble radius today were within the comoving Hubble radius at some point (and were thus within the comoving horizon as well).  For the comoving Hubble radius to decrease, $aH$ must increase, \emph{i.e.}
$$\frac{d}{dt}(aH)=\frac{d}{dt}\left(a\cdot\frac{da/dt}{a}\right)=\frac{d^2a}{dt^2}>0\;.$$
This describes what in cosmology is called inflation.  According to the inflation model, the universe underwent a period of very rapid growth.  Dodelson argues that given that inflation is usually thought of to operate at $T\sim10^{15}$ GeV, this implies that the comoving Hubble radius at the end of inflation was $10^{-28}$ times that of today.  For the inflation model to help solve the horizon problem, it must be that the comoving Hubble radius right before inflation was much larger than today's value.  
\section*{Inflation As A Solution}
It turns out that cosmic inflation helps to solve many of these problems.  Recall that the horizon problem: it didn't seem conceivable for photons in the CMB from opposite ends of the universe to be in thermal equilibrium if they are just being observed now.  This would seem to imply that they were never able to interact in the past in order to come into equilibrium in the first place.  Since before inflation, the comoving Hubble radius was much larger than it is today, photons were able to achieve thermal equilibrium before and during inflation.  The fact that the comoving Hubble radius is increasing now, therefore, does not imply that it was never higher than it is now.\\
\begin{figure}[h]
\includegraphics[width=0.5\textwidth]{dodelson65cropped}
\caption{This is figure 6.5 from \cite{Dodelson}.  This should be compared to figure 1 above.  The larger cone in this figure relative to figure 1 represents the horizon in an inflationary model.  Since much more of the universe was in causal contact before and during inflation, it is now possible for the CMB photons to be in thermal equilibrium before they reach Earth.}
\end{figure}
\indent Inflation is able to say something about the flatness problem as well.  Recall that since the right hand side of equation \ref{eq:flatmain} is constant, the left hand side must be as well.  The density of the field that causes inflation is approximately constant during inflation \cite{wiki:flatness}, but the scale factor $a$ increases very rapidly over this period.  This implies that the quantity $\left(1-\frac{1}{\Omega}\right)$ must decrease by this much.  Given the fact that $a$ increases exponentially over this period, it's not inconceivable that this quantity should diminish to $\sim10^{-62}$. \\
\indent Regarding the fact that there seemingly should be a high density of magnetic monopoles, yet we don't observe any, inflation is able to explain this as well.  If $a$ increases exponentially at a temperature higher than the temperature at which monopoles are created, then the density of monopoles surely decreases by a dramatic amount with respect to what it would have been without inflation \cite{wiki:inflation}.  As such, it is not surprising that monopoles have yet to be observed.  Perturbations in the early universe are created by quantum fluctuations in the scalar field that causes inflation.  In terms of explaining why $\Lambda$ is not equal to zero, I would say that inflation provides a partial explanation.  As discussed above, inflation provides a plausible explanation for why the density of the early universe was tuned to the value needed in order to expand into the universe as we know it today.  The universe today is very close to (if not exactly) flat, meaning that the total density of the universe should be very close to (if not equal to) the critical density.  We know that $\Lambda$ is non-zero from studies of high redshift Type Ia supernovae, and we know the universe is nearly flat (brought on due to inflation).  What ties these concepts together is dark energy.  The presence of dark energy in the universe is what is thought to cause the accelerated expansion of the universe, and it also makes $\rho/\rho_c\sim 1$.  Inflation, while it sets the stage for the universe to be nearly flat, does not explain what dark energy actually is.
\bibliographystyle{unsrt}
\bibliography{ps3Sources}
\end{document}