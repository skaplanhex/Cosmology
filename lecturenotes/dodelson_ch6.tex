\documentclass{article}
\usepackage{fullpage,amsmath,amssymb,accents}
\title{Notes on Dodelson Ch. 6}
\author{Steven Kaplan}
\date{February 24-25, 2015}
\begin{document}
\maketitle
Feedback on problem set 1:
\begin{itemize}
\item big crunch in 80 Gyr, big rip in 33 Gyr
\item What happens with curvature?\\ \\
curvature term: $\frac{k}{a_0^2}\left(\frac{a}{a_0}\right)^{-2}$.  We're not free to specify the scale factor in a curved universe!  How to gain that freedom?  Solve for today: $k=a_0^2H_0^2\left(\sum\Omega_i-1\right) \equiv a_0^2H_0^2(-\Omega_k)$.  Thus, the curvature term in the Friedmann equation is $-H_0^2\Omega_{k,0}$.
\end{itemize}
Begin discussion of the first half of Dodelson chapter 6.  Simplified initial conditions: (6.1,6.2).  The velocities all $\rightarrow 0$.  Further approximations lead to 6.8.  Read the rest of section 6.1 for the rest of the development.
\\ \\
6.3: solution to horizon problem:
\\ \\
Inflation = rapid, exponential expansion (i.e. negative pressure $w\sim-1$ caused by (for example) a scalar field as in figure 6.3, physical hubble radius constant, comoving hubble radius (apparent horizon) $(aH)^{-1}$ decreasing
\\ \\
region that was in causal contact $\rightarrow$ outside apparent horizon.  Decreasing $(aH)^{-1}$ requires $\ddot{a}>0$ (6.18).  {\emph i.e.} acceleration, $w<-1/3$ (6.23).  Comoving horizon had to decrease by $10^{28}$, {\emph i.e}. about 60 $e$-folds.
\\ \\
Fig 6.5, horizon problem solved.  Redefine $\eta$
$$\eta=\int_{t_{\mathrm{end\;of\;inflation}}}^{t}\frac{dt'}{a(t')}$$
\S 6.3.3: scalar field $\phi(\vec{x},t)$, use homogeneous part $\phi^{(0)}(t)$.  Start from E-M tensor (6.24) for $\phi\rightarrow$(6.26).  Example: free field with mass $m$ has
$$V(\phi)=\frac{m^2\phi^2}{2}$$
like a particle in a potential well with $x\rightarrow\phi$.  (6.27), hence false vacuum (Fig 6.6) has $p<0$ and $w\simeq-1$.  False vacuum is 'old' inflation, dismissed because true vacuum never reached except in small bubbles, so inflation is \underline{eternal}.  'New' inflation = 'slow roll' down $V(\phi)$, if slow enough, $V$ dominates such that $p<-1/3$, acceleration.  This leads to equation (6.33).  Note that overdots are defined as $\cdot \equiv d/d\eta$.  Define $\epsilon$ as in (6.35) and $\delta$ as in (6.36).  $H<0\rightarrow \epsilon>0$, inflation requires $\epsilon<1$.  $\delta$ quantifies the rate of the slow roll and should not be confused with overdensity.
\newpage
\textbf{Start Feb. 25 notes:}
\\ \\
Clarification: the 9 perturbation variables are: 
\begin{enumerate}
\item $\Phi$, curvature perturbations.
\item  $\Psi$, newtonian potential
\item $\Theta$, distribution function for photons
\item $\eta$, distribution function for neutrinos
\item $\delta$, distribution for dark matter
\item $\delta_b$, overdensity
\item $v$
\item $v_b$
\item $\Theta_p$ 
\end{enumerate}
Start with \S 6.4: gravitational wave production.  Gravitational wave production due to QM tensor fluctuations in metric.  From $h_+,h_\mathrm{x}$ that satisfy (6.45), quantize harmonic oscillator to produce (6.53).  $P_h(k)$ is a power spectrum and is scale free since it is $\propto k^{-3}$, \emph{i.e.} power law.  The form of $P_h(k)$ is given in (6.59).  $H$ in the expression is evaluated at the time of the horizon crossing.  Look at figure 6.7.  It shows the amplitude drops until the wave vector $k$ becomes super-horizon and then becomes constant.
\\ \\
$\therefore$ detecting gravitational waves (GWs) measures $H\sim$ 60 $e$-folds before end of inflation which in turn measures $V(\phi)$ at $\gtrsim 10^{15}$ GeV. (compare to 14 TeV LHC, not yet detectable!)  This is iff inflation occurred at Planck scale.  GW \& scalar perturbations generically predicted to be gaussian $\equiv$ 2$^{nd}$ moment describes distribution completely.
\\ \\
\S 6.5: scalar perturbations $\Psi$
\\ \\
We decompose the scalar field as in 6.60:
$$\phi(\vec{x},t) = \phi^{(0)}(t)+\delta\phi(\vec{x},t)$$
$\longrightarrow$(6.71).  Just like tensors, $P_{\delta\phi}=\frac{H^2}{2k^3}$.  Figure 6.8: can form $\zeta$ as in (6.77).  $\zeta$ is conserved after the horizon crossing.  See (6.79) and (6.80) and their explanations.  Then see development to (6.82).  We see that scalar to tensor ratio is on the order of $1/\epsilon$.  Remember that $\epsilon$ is bound by 0 and 1 for slow roll inflation.  More useful forms: (6.35,6.36).
\\ \\
\S 6.6:
\\ \\
The average perturbation over the universe $\left<\Phi(\vec{k})\right>$ is 0 by definition (6.98).  The second moment is given by (6.99).  $P_\Phi(k)$ is the primordial power spectrum.  Inflation predicts $k^3P_\Phi(k)$ is about constant i.e. scale invariant but so did aesthetic arguments of Harrison-Zel'dovich-Peebles in the 1970s.
\\ \\
Signatures of inflation:
\begin{itemize}
\item small mixture of tensor modes (iff at $\sim 10^15$ GeV)
\item small deviations from scale invariance due to slow roll
\end{itemize}
See (6.100) and accompanying explanation.  $n=1,\;n_T=0\equiv$ scale invariant.  See (6.104) and (6.106).

\end{document}