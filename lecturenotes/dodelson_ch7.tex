\documentclass[12pt]{article}
\usepackage{fullpage,amsmath,amssymb}
\title{Notes on Dodelson Ch. 7}
\author{Steven Kaplan}
\date{March 3, 2015}
\begin{document}
\maketitle
\subsection*{Pressure vs. diffusion}
Pressure $P=-\frac{dE}{dV}$.  It can be thought of as resistance to being squeezed.  It involves collisions and can serve to counteract gravity.
\\ \\
Diffusion is the result of a density gradient and thermal/random motions.  Requires neither gravity nor collisions.
\subsection*{\S 7.1}
For dark matter perturbations, "simple" at early times because it is radiation perturbations (overdense) (at early times) that set the potential wells.  The opposite is true in undermines regions (set potential 'bumps').  At late times, this is independent of radiation due to matter domination (MD).
\\ \\
Gravitational Instability - $\delta\sim10^{-5}$ in early times $\longrightarrow \delta\sim 1 $ in late times.  Non-linearity $\rightarrow$ collapse.  Equation (7.1): $\ddot{\delta}+[\mathrm{Pressure}-\mathrm{Gravity}]=0$.  If pressure $\rightarrow$ 0, $\delta=e^{kt}+e^{-kt}$ where the $e^{-kt}$ term goes to zero.  If pressure $>>$ gravity, $\delta \propto \cos(kt)$.  
\\ \\
Now look at figure 7.1.  Equation (7.2) defines the field of large scale structure.  Figure 7.2.
\\ \\
Radiation domination (RD) ($a<<a_{eq}$): $\Phi$ drops once a mode enters the horizon.  MD ($a >>a_{eq}$): $\Phi$ is constant.  Caveat: until dark energy becomes important, at a redshift $z\simeq 1$.  Dark energy causes a decay of the perturbations.  Transfer function($k$) gets us from inflation to full MD.  Equation (7.5) is a mathematical representation of (7.2).  The factor of 9/10 can be seen in (7.3) and in the top curve of figure 2.  $D_1(a)$ is the growth function of $\delta$ (not for $\Phi$).  Iff MD, $D_1(a)=a$.  
\\ \\
Look at figure 3.  Equation (7.6) Poisson equation relates $\delta$ to $\Phi$.  Applies when $a>>a_{eq}$.  Sub in for $\rho_m$, solve for $\delta$ and together with (7.5) you get (7.8).  The power spectrum is given by (7.9).  It has the dimensions of length$^3$.  We'd like a dimensionless version $k^3P(k)$.  Look at figure 7.4.  At large scales, $T\rightarrow 1$, so $P(k)\propto k$.  Turnover (at horizon scale of M-R equality) explained by figure 7.2 decline = retardation of growth on small scales fig 7.3 (c.f. $k^{3/2}$).  $\Lambda$CDM has less M, later equality, larger horizon scale at eq.  Figure 7.5 is less important.
\\ \\
\subsection*{\S 7.2}
Large Scale Perturbations, super-horizon (i.e. far outside the horizon) $k\eta<<1$.  The horizon for homogeneity is the real horizon, but the horizon for perturbations is the "apparent" one.  Think about this.  Equation (7.21) defines $y\equiv\frac{a}{a_{eq}}=\frac{\rho_{dm}}{\rho_r}$.  This leads to (7.32).  Assume that $y$ is small ($<<1$), then $\Phi=\Phi(0)$.  By definition, small $y$ is in the RD regime.  In the large $y$ case (post-equality), $\Phi=\frac{9}{10}\phi(0)$.
\\ \\
Look at figure 7.6.  Shows the approximation vs. the full solution.  In \S 7.2.2, Dodelson shows that $\dot{\Phi}=0$, i.e. the potential is constant, is also a solution during MD.  Finish reading \S 7.2.2.
\subsection*{\S7.5}
Growth function - independent of $k$ (wave number, spatial scale) (a la Meszaros, eqn (7.58)).  At late times, all relevant scales are sub-horizon.  If $\Omega_{m,0}\neq 1$, need to account for effects on growth D.E. and/or curvature.  Section 7.5 is Dodelson working through this.  Equation 7.77 is the 'official' solution to this.  Growth will \underline{still} be independent of $k$ because only radiation smooths out inhomogeneities.  Figure 7.12 shows solutions to 7.77.
\\ \\
Baryons are 4\% of $\rho_{c,0}$\\
Neutrinos are $\sim0.1\%$ of $\rho_{c,0}$.  Suppresses $T(k)$ at large $k$.\\
Dark Energy is 70\% of $\rho_{c,0}$.  Suppresses growth factor.\\
\end{document}